\documentclass[10pt,a4paper]{article}

\usepackage{ifpdf}
\ifpdf 
    \usepackage[pdftex]{graphicx}   % to include graphics
    \pdfcompresslevel=9 
    \usepackage[pdftex,     % sets up hyperref to use pdftex driver
            plainpages=false,   % allows page i and 1 to exist in the same document
            breaklinks=true,    % link texts can be broken at the end of line
            colorlinks=true,
            pdftitle=The Mitten Programming Language
            pdfauthor=Oliver Katz
           ]{hyperref} 
    \usepackage{thumbpdf}
\else 
    \usepackage{graphicx}       % to include graphics
    \usepackage{hyperref}       % to simplify the use of \href
\fi

\usepackage[margin=1.0in]{geometry}

\title{The Mitten Programming Language}
\author{Oliver Katz}
\date{\today}

\setcounter{tocdepth}{2}

\begin{document}
\maketitle
\tableofcontents
\newpage

\section{Introduction}
This is the standard of the Mitten language. This standard has different versions, the naming of which is described in section \ref{sec:Versioning}. The semantic syntax is described in EBNF, a summary of which is given in section \ref{sec:EBNFSummary}. For a formal definition, see \textit{XXX}.

Mitten is intended to be a high-level programming language for large projects. It is multi-paradigm, encompassing object-oriented, procedural, and functional programming. It is designed to be run natively, although this is left to the implementation's discretion. If you already know the C programming language, or especially C++, the syntax and functionality of Mitten should be familiar. This document is a description of Mitten for those who want to write code in it and for those who want to write a compiler for it.

This document is a technical specification of the Mitten language. It provides standardization for compilers so that any Mitten code should work the same with any compiler implementation. It contains a description of the lexical grammar, the syntax, the preprocessor, the expected execution of a given syntax, and the most basic standard library necessary for Mitten.

The best introduction for the programming language is a simple example. The following code prints out the message ``hello, world." to the console:
\begin{verbatim}
include(std);

void main()
{
    print("hello, world\n");
}
\end{verbatim}

This is one of the simplest programs that can be written in Mitten. It will be explained in detail in the beginner tutorial found in the next section.

\subsection{Comparison to Other Languages}
Mitten takes much inspiration from other existing programming languages. Its syntax is notably C-like, categorizing Mitten as a C-based programming language. Its object orientation system is based heavily off C++, although there are distinct differences. It's dynamicy is very similar to that of Python, although limited by the fact that Mitten was designed to run natively. The functional programming features implemented with Mitten are inspired by Haskell's ability to utilize functions efficiently, although this system is much simpler in Mitten.

Although Mitten can be thought of as an extension to C, it is not legacy compatible with the legendary programming language. The main difference, besides the extension, is the preprocessor. Mitten's preprocessor is based partially off C's, but the syntax is different; it does not use the \verb|#define x 5| directive syntax, but instead uses the same syntax as the rest of the language: \verb|define(x, 5);|. This is to allow a single pass of the text for compilation, instead of two: one for preprocessing, one for compiling.

C++'s object-orientation system is incredibly comprehensive, but there are some things that are different between C++ and Mitten that make Mitten's syntax simpler. The most profound difference is in how polymorphism works. In Mitten, assume you have two classes \verb|Superclass| and \verb|Subclass| (\verb|Subclass| obviously inherits from \verb|Superclass|) and create an object \verb|o| of \verb|Subclass|. If you pass \verb|o| to a function that takes an argument of type \verb|Superclass| and call a method overloaded by \verb|Subclass|, \verb|Subclass|'s method will be called, not \verb|Superclass|'s. This profound difference allows for casting and polymorphism workarounds to be less heavily used in Mitten. The other significant difference is templates. C++'s template syntax is bulky and sometimes difficult to read, while in Mitten it is kept simple. Additionally, Mitten provides an \verb|auto| type for quick generic methods. Every time you call a method with an \verb|auto| argument, a new ``version" of that function is generated with the correct type. The \verb|auto| type is truly automatic.

Python provides the \verb|dir| function, invaluable for dynamic programs. It returns a dictionary of all the members and methods of any class, module, method, or object within the programming language. Mitten provides a similar functionality with the \verb|table| function. Every time the \verb|table| function is called on any symbol, a tabular dictionary with all the members and methods of that symbol is created and returned. For native implementations, this will require extra memory used, but since it can be lazily generated for only the symbols that it is required for the overhead can be quite small. The disadvantage of Mitten's \verb|table| to Python's \verb|dir| is that modification of the dictionary returned by Python's \verb|dir| results in modification of the runtime object that it represents. Mitten does not have this functionality with \verb|table|, thus the name difference.

Haskell is remarkable in its syntax effiency. Every function is written on a single line and there are no loops, these restrictions are easily counteracted by the extraordinarily powerful function system within the language. Currying, higher-order functions, and a functional type system are the basis of this system. Mitten provides the ability for function calls to curry functions when explicitly specified, allowing for optimizations. Functions, curried or not, may be passed as variables, allowing for higher-order functions and have a strict type system allowing for safety within the runtime. Mitten extends this system by providing method callbacks, pairing the function itself with the object it belongs to.

Although these languages cannot be matched for their specific advantages, Mitten does take heavy inspiration from them. C, C++, Python, and Haskell are all extraordinary languages to learn, as they should be.

\subsection{Versioning}
\label{sec:Versioning}
Since Mitten has an official compiler (the Mitten Compiler MC), the standard is given the same version number as the current compiler version. For example, the current compiler version for this standard is 0.01-alpha, which means that this standard will also be numbered 0.01-alpha for its correlation with this compiler version. It is expected that the compiler will update more often than the standard, which is why it is the basis for the versioning system.

The following table lists the history of the Mitten language standard and Mitten Compiler.
\begin{tabular}{|l|l|l|l|}
\hline
\textbf{Language Standard} & \textbf{Compiler} & \textbf{Date} & \textbf{Primary Change} \\
\hline
0.01-alpha & 0.01-alpha & August 11, 2014 & Initial release. \\
\hline
\end{tabular}

\subsection{Terminology}
The following terms are used in the specification:
\begin{description}
\item[Token] A sequence of characters making up the smallest unit of lexical syntax.
\item[Deliminator] A pattern or sequence of characters used to separate non-deliminator tokens in lexical analysis.
\item[Symbol] A named element of the syntax for the given language. 
\item[Terminal Symbol] An elementary syntactic symbol which is atomic in the syntax of the given language.
\item[Nonterminal Symbol] A symbol which is comprised of other terminal and nonterminal symbols in the syntax of the given language. Their specification often takes the form of a production rule.
\end{description}

\subsection{EBNF Summary}
\label{sec:EBNFSummary}
Extended Backus-Naur Form (EBNF) is a formal syntax for specifying programming language syntax. Its syntax consists of terminal symbols and production rules. The simplest production rule states that a nonterminal symbol consists of a terminal symbol:
\begin{verbatim}
zero = "0";
\end{verbatim}

A nonterminal symbol can be any of a set of possible terminal symbols:
\begin{verbatim}
digit = "0" | "1" | "2" | "3" | "4" | "5" | "6" | "7" | "8" | "9";
\end{verbatim}

It can also be a combination of terminal and nonterminal symbols:
\begin{verbatim}
hex digit = digit | "a" | "A" | "b" | "B" | "c" | "C" | "d" | "D" | "e" | "E" | "f" | "F";
\end{verbatim}

You can specify symbols that go in sequence:
\begin{verbatim}
hi = "hi"; (*this is the same as...*)
hi = "h", "i"; (*this.*)
\end{verbatim}

It is also easy to specify an element which can be zero or more instances of a symbol in sequence:
\begin{verbatim}
hex number = "0x", hex digit, {hex digit};
\end{verbatim}

Or an element which is optional:
\begin{verbatim}
might be a question = "dinner", ["?"];
\end{verbatim}

Parenthesis work much as they do in algebraic expressions:
\begin{verbatim}
binary string = ("1", "0"), {"1" | "0"};
\end{verbatim}

This is a basic summary of EBNF syntax to refresh the reader before reading the specification. For a more detailed description, see \textit{XXX}. Note that all EBNF syntax in this document uses ASCII unless otherwise specified.

\section{The Lexical Grammar}
The lexical grammar of Mitten is used by both the preprocessor and the language itself. It will be described as a set of deliminators. Deliminators may be set sequences of characters or character sequence patterns which will be detailed below. All nonterminal symbols listed in the subsections below are considered deliminators with the given names.

\subsection{Input Data Format}
The lexical grammar provides certain requirements on the input data. Mitten source text can be of any of the following encodings:
\begin{itemize}
\item 7-bit ASCII
\item UTF-8
\item UTF-16BE
\item UTF-16LE
\item UTF-32BE
\item UTF-32LE
\end{itemize}

The encoding of a given text file may be identified by a UTF BOM (Byte Order Marking) at the beginning of the file:
\begin{tabular}{|l|l|}
\hline
\textbf{Encoding} & \textbf{BOM} \\
\hline
UTF-8 & EF BB BF \\
\hline
UTF-16BE & FE FF \\
\hline
UTF-16LE & FF FE \\
\hline
UTF-32BE & 00 00 FE FF \\
\hline
UTF-32LE & FF FE 00 00 \\
\hline
7-bit ASCII & (no BOM) \\
\hline
\end{tabular}

If a text file does not have a BOM, the first character must be less than or equal to 7F, meaning that it is 7-bit ASCII. Since UTF encodings are supersets of 7-bit ASCII, 7-bit ASCII codes may work as codes for all other encodings listed here.

\subsubsection{Note on Internationalization}
Other encodings may be used for internationalization as long as they conform to the internationalization requirements listed in section \ref{sec:Internationalization} below.

\subsection{Whitespace Characters}
\begin{verbatim}
space = " ";
horizontal tab = "\t";
newline = "\n";
whitespace = space | horizontal tab | newline;
\end{verbatim}

\subsection{Punctuation Tokens}
\begin{verbatim}
open expression = "(";
close expression = ")";
separate expression = ",";

open scope = "{";
close scope = "}";
separate scope = ";";

open subscript = "[";
close supscript = "]";

quote = "\", ?any printable characters and escape codes?, "\";
character quote = "'", ?any single printable character or escape code?, "'";
block quote = "\"\"\"", ?any characters?, "\"\"\"";

line comment = "//", ?any characters?, newline;
block comment = "/*", ?any characters?, "*/";
\end{verbatim}

\subsection{Operators}
\begin{verbatim}
add = "+";
subtract = "-";
multiply = "*";
divide = "/";
modulate = "\%";
bit and = "&";
bit or = "|";
bit xor = "^";
bit shift left = "<<";
bit shift right = ">>";
bit negate = "~";
logical and = "&&";
logical or = "||";
logical xor = "^^";
logical negate = "!";
assign = "=";
add assign = "+=";
subtract assign = "-=";
multiply assign = "*=";
divide assign = "/=";
modulate assign = "\%=";
bit and assign = "&=";
bit or assign = "|=";
bit xor assign = "^=";
bit shift left assign = "<<=";
bit shift right assign = ">>=";
bit negate assign = "~=";
equal to = "==";
not equal to = "!=";
less than = "<";
less than or equal to = "<=";
greater than = ">";
greater than or equal to = ">=";
range = ":";
\end{verbatim}

\section{Semantic Grammar}
There are certain syntactic elements which are repeated heavily throughout Mitten. They are listed below. All nonterminal symbols listed in the subsections below are considered semantic gramatical elements.

\subsection{Expressions}
\begin{verbatim}
expression = open expression, [whitespace], [expression element, [whitespace], {expression separator, [whitespace], expression element}], [whitespace], close expression;

expression element = expression | ...;
\end{verbatim}

\subsection{Scopes}
\begin{verbatim}
scope = (open scope, [whitespace], [scope element, [whitespace], {scope separator, [whitespace], scope element}], [whitespace], close scope) | scope element;

scope element = expression | scope | ...;
\end{verbatim}

\subsection{Internationalization}
Mitten, as a programming language, supports a multitude of other languages through their character encodings. Mitten compilers must comply with the following requirements in order to support internationalization:
\begin{itemize}
\item Mitten compilers can support 1, 2, and 4 byte character encodings using input file's BOMs for encoding detection as well as a runtime option to specify a source file's encoding.
\item Mitten compilers can read and fully utilize internationalization packages in the format given by this specification on runtime for a given source file.
\end{itemize}

The internationalization package mentioned in these requirements is a list of all built-in symbols in the language's syntax mapped to the default symbol set. The file is designed so that it does not rely on any intersection with 7-bit ASCII in its syntax. Internationalization packages for Mitten compilers will have the following extension: \verb|*.mip|. 

The first byte of the file is an unsigned integer which represents the character width of the encoding used in by the language's required character encoding. For example, a language using 7-bit ASCII would specify 1, while a language using a 4-byte character encoding would specify 4.

The second byte is of the file is an unsigned integer which represents the length of the BOM in bytes to look for. If the text file is not expected to use a BOM, this byte should be 0. The next n bytes of the file will contain the expected BOM where n is the length of the BOM specified by the second byte of the file.

For example, for UTF-32BE, The first 6 byte of the internationalization package would be: 04 04 00 00 FE FF.

The next n bytes of the file, where n is the character width given by the first byte of the file, represents the separator character of the file which will be used in the remainder of the file. This completes the header of the package.

For a 7-bit ASCII language with a newline separator character, the file header would be: 01 00 0A.

\section{The Preprocessor}

\end{document}
